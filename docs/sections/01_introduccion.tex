\section{Introducción}
El análisis de multiples objetos acoplados es recurrente para distintas aplicaciones de control o mantenimiento. Su diseño parte de un modelo matemático que puede describir su comportamiento en el tiempo. Particularmente, el sistema de osciladores armónicos oscilatorios también pueden describirse apropiadamente por su frecuencia. A partir de su dinámica dada por la ecuación clásica \ref{eq:model}, se puede formular una solución análitica.

\begin{equation}
	\begin{split}
		m \ddot x_i & = -D(x_i - x_{i+1}) - D(x_i - x_{i-1}) \\
		& = -D(2x_i - x_{i+1} - x_{i-1})         
	\end{split}
	\label{eq:model}
\end{equation}

De la ecuación \ref{eq:model}, $x_i$ representa a la desviación de la partícula $i$ de su posición de reposo que oscila acoplada entre $n$ partículas, $D$ es la constante de elasticidad de todos los resortes que unen cada partícula. Se plantea que los extremos de la cadena están fijos a $x_0 = x_n = $ constante, con las relaciones representadas en el sistema de ecuaciones \ref{eq:state-equation}

\begin{equation}
	\begin{split}
		\sqrt{m} x_i(t) &= v_i \exp(i \omega t) 
		\\
		y_i &= \sqrt{m_i}x_i 
	\end{split}
	\label{eq:state-equation}
\end{equation}

Actualmente, se usa para el control o simulación de estos modelos computadores digitales que mediante distintas variaciones de procesos, puede incrementarse el desempeño. En este trabajo, se plantea resolver el problema utilizando paralelismo mediante OpenMP. Para ello, se arreglará el sistema de ecuaciones para su resolución matricial. Además de ello, se mostrará un análisis del desempeño del programa respecto al tamaño $n$ del sistema y el número de hilos utilizados. Finalmente, se discuten los patrones de los resultados y su implicancia para la escalabilidad.



