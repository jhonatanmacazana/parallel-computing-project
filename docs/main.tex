\documentclass[conference]{IEEEtran}
\IEEEoverridecommandlockouts
%\usepackage{hyperref}
\usepackage{cite}
\usepackage{amsmath,amssymb,amsfonts}
\usepackage{algorithmic}
\usepackage{graphicx}
\usepackage{textcomp}
\usepackage{xcolor}
\usepackage{import}
\usepackage[utf8]{inputenc}
\def\BibTeX{{\rm B\kern-.05em{\sc i\kern-.025em b}\kern-.08em
    T\kern-.1667em\lower.7ex\hbox{E}\kern-.125emX}}
\graphicspath{ {images/} }
% BUG in bibtex inside IEEEtran for bibliography
\makeatletter
\def\endthebibliography{%
  \def\@noitemerr{\@latex@warning{Empty `thebibliography' environment}}%
  \endlist
}
\makeatother

% BUG in amsmath for equations
\makeatletter
\renewcommand{\dddot}[1]{%
  {\mathop{\kern\z@#1}\limits^{\vbox to-1.4\ex@{\kern-\tw@\ex@
   \hbox{\normalfont ...}\vss}}}}
\renewcommand{\ddddot}[1]{%
  {\mathop{\kern\z@#1}\limits^{\vbox to-1.4\ex@{\kern-\tw@\ex@
   \hbox{\normalfont....}\vss}}}}
\makeatother

\begin{document}

\title{Oscilador Armónico Acoplado}

\author{
	\IEEEauthorblockN{Jhonatan Macazana and Samir Muñoz}
	\IEEEauthorblockA{
		\textit{Departamento de Ingeniería Electrónica} \\
		\textit{Universidad de Ingeniería y Tecnología}\\
		Lima, Perú \\
		\{jhonatan.macazana \& emanuel.munoz\}@utec.edu.pe
	}
}
	
\maketitle
	
\begin{abstract}
	Los modelos armónicos representan un reto para su análisis, pero es beneficioso su solución para distintas aplicaciones. En el presente trabajo, se presenta la implementación y análisis de un sistema oscilador armónico acoplado. Se propone además una paralelización usando OpenMP utilizando convenientemente una de las soluciones conocidas matricialmente. Tambien, se específican los modelos matemáticos como también particularidades del código utilizado. Finalmente, se discuten los resultados gráficos en distintos dispositivos de tiempo y desempeño.
\end{abstract}
	
\begin{IEEEkeywords}
	oscilador, OpenMP, desempeño
\end{IEEEkeywords}
	
\import{sections/}{01_introduccion.tex}
	
\import{sections/}{02_metodologia.tex}
	
\import{sections/}{03_resultados.tex}
	
\import{sections/}{04_conclusiones.tex}

	
%% ============================================================================
\renewcommand{\refname}{Referencias}
\bibliographystyle{IEEEtran}
\bibliography{references}
	
\end{document}
